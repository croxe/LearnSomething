\documentclass[12pt]{article}
\begin{document}
\section{Android Debug Bridge}
\begin{verbatim}
Command line through the terminal
Default Path: "Username"\AppData\Local\Android\Sdk\platform-tools\adb.exe

1. adb devices
Find all devices currently connected

2. adb logcat
Print log information in the devices
adb logcat -c 				Clear buffer log
adb logcat -b <buffer> 		Print specific log information
adb logcat -v <format>	 	Print log in format
adb logcat -f <file> 		Print to specific file path

3. adb install/uninstall
Install or uninstall apps
Ex: adb install ~/Downloads/Facebook.apk
Ex: adb uninstall com.facebook.facebook

4. adb pull/push
Copy and paste between local and devices
Ex: adb push ~/temp/GithuApp.trace /sdcard
Ex: adb pull /sdcard/GithubApp.trace ~/

5. adb start/kill-server

6. adb shell
start Android shell

7. adb connect/disconnect
connect/disconnect wifi

8. adb shell am
"am" is the Android Activity Manager
 
9. adb shell pm
"pm" is the Android Package Manager

10. adb shell dumpsys
Output system information

11. adb tcpip
listen to specific port
Ex: adb tcpip 5555 "listen to 5555 port"

Emulator Command:

12. emulator -list-avds
List all devices available
\end{verbatim}
\section{Android Manifest}

Android Manifest is a xml, which contains data for user interface. Every android app must have a manifest with same name and located at the root of the package. More importantly 
it reflect important feature of the app.

\begin{verbatim}
Under <manifest><application>
<activity> for each subclass of Activity.
<service> for each subclass of Service.
<receiver> for each subclass of BroadcastReceiver.
<provider> for each subclass of ContentProvider.

Manifest and activity can be named.
<manifest package = "com.example.myapp" ... > 
<activity android:name = ".MainActivity" ... >

When App need access to sensitive data, it needs Android Permission.
<uses-permission android:name="android.permission.SEND_SMS"/>

The <uses-feature> element allows you to declare hardware and software 
features your app needs.
<uses-feature android:name="android.hardware.sensor.compass"
              android:required="true" />

<uses-sdk> allows you to declare the minimum sdk requirement.
\end{verbatim}




\end{document}


